%%%%%%%%%%%%%%%%%%%%%%%%%%%%%%%%%%%%%%%%%%%%%%%
% latex_template_ieee_bbing.tex
%
% 28.02.2008  bsb Created 
%%%%%%%%%%%%%%%%%%%%%%%%%%%%%%%%%%%%%%%%%%%%%%%%%

% Final - 2 column style
%\documentclass[10pt,final,journal]{../latexlib/latex_ieee/IEEEtran}
% Draft - single column style
\documentclass[11pt,draftcls,journal,onecolumn]{../latexlib/latex_ieee/IEEEtran}

% If the IEEEtran.cls has not been installed into the LaTeX system files, 
% manually specify the path to it:
% \documentclass[conference]{../sty/IEEEtran} 

% Gives LaTeX2e the abilsity to do double column
%% correct bad hyphenation here
\hyphenation{op-tical net-works semi-conduc-tor IEEEtran}

% From thesis main (bbing)
\usepackage{amssymb,longtable,dcolumn}

% to use pdflatex
% Standard bbing packages
\usepackage{cite}      % Written by Donald Arseneau
\usepackage{graphicx}  % Written by David Carlisle and Sebastian Rahtz
\usepackage{url}       % Written by Donald Arseneau
\usepackage{amssymb,longtable,dcolumn}
\usepackage{bm}
%\usepackage{subfigure}
%\usepackage{stfloats}  % Written by Sigitas Tolusis
\usepackage[caption=false,font=footnotesize]{subfig}
%\usepackage{fixltx2e}
\usepackage{colortbl}
\usepackage{multirow}
\usepackage{amsmath}
\usepackage{units}
\usepackage{../latexlib/latex_cmds/math_bbing}
\usepackage{acronym}
\usepackage{csvsimple}
\usepackage{../latexlib/latex_cmds/my_acronyms}
\usepackage{color,soul}

\begin{document}

\newtheorem{remark}{Remark}
\renewcommand{\theremark}{\unskip}


\input{../latexlib/latex_cmds/Commands}  % shortcuts to thesis stuff
\input{../latexlib/latex_cmds/defs}
%\include{./latex_cmds/Commands}  % shortcuts to thesis stuff
%\include{./latex_cmds/defs}


% set the figure default size
\newcommand{\SF}{0.2}
\newcommand{\SFb}{0.45}
\newcommand{\SFPic}{0.45}
\newcommand{\SFPlot}{0.45}
\newcommand{\SFc}{0.25}
% Just a lazy way of setting the figure width (percentage of text width)
% 0.7 works well for 1 column
% 0.4 works well for 2 column
\newcommand{\FigWidth}{\SFb}

% Use this one for the draft version
\newcommand{\scaleOneTwo}[2] {\scalebox{#1}}
% Use this one for the two column version
%\newcommand{\scaleOneTwo}[2] {\scalebox{#2}}

% Graphics for this paper
\graphicspath{{./figs/}}

% paper title
%\title{Towards an Experimentally Validated Plume Model to Support Robotic Plume Characterization}
\title{USV Modeling}

% author names and affiliations
% use a multiple column layout for up to three different
% affiliations
\author{Brian~Bingham$^{1}$% <-this % stops a space
\thanks{$^{1}$ Brian~Bingham is with the Department of Mechanical and Aerospace Engineering, Naval Postgraduate School, Monterey, CA 93950, USA. {\tt\small bbingham@nps.edu}}%
}

% make the title area
\maketitle

\begin{abstract}
Abstract
\end{abstract}
% no keywords

\IEEEpeerreviewmaketitle

\section{Introduction}
The goal of this effort is to develop a USV model that 
\begin{itemize}
\item Makes use of physically measurable quantities (mass and inertia)
\item retains fidelity appropriate for 
\item with predictive capabilities with and without water current
\item that can be identified using static tests and manuevering trials with a USV and onboard sensors
\end{itemize} 

\section{Background}

\subsection{\cite{sonnenburg10control} and \cite{sonnenburg13modeling}}

\cite{sonnenburg13modeling} and \cite{sonnenburg10control} examine model for USV with steerable outboard motor (vectored thrust) where sideslip is a major concern.  Uses notation and maneuvering model from \cite{fossen94guidance}.
\begin{itemize}
\item Full vessel model is includes linear and quadratic damping
\item All models are then linearized (perturbation dynamics) for the purpose of identification where the coefficients are parameterized by the states (surge, sway and yaw-rate)
\item Actuation model thrust as a linear  and quadratic with velocity depedence.
\end{itemize}
Model identification appears to throw out the phyical model and rely on identification of linear models (perturbation models) and a set of discrete speeds.
\begin{itemize}
\item Model idetification includes open-loop manuevers to identify steady-state parameters and closed-loop maneuvers to identify dynamic parameters.
\begin{enumerate}
\item Steady-state values are identified by as linear relationships between yaw-rate, side-slip angle and side-sleep speed.  The coefficients of these a relationships are determined for a set of discrete, constant speeds.  The values of these coefficients change significantly over the range of forward speeds and now functional relationship is offered.  
\item Thurster model conflates both the thrust relationship and the inertia of the system.  Model is identified by measuring initial acceleration during step changes in engine RPM and identifying the two linear coefficients of a linear model.  This single model is constant over the range of speeds.  Sparce data at higher speeds.
\item Speed/surge is modeled as first-order model parameterized over speed range.
\item Steering model (first order Nomoto model with sideslip) is indentified by minimizing a quadratic cost function of sideslip angle and yaw-rate over close-loop time histories. Again, the linearized models have coefficients that change with speed.  The results again show significant changes over the speed envelope without offering a functional relationship.
\end{enumerate}
\end{itemize}

\subsection{\cite{caccia08practical}}
Uses the nonlinear model of Blanke \cite{blanke81ship} as reported in \cite{fossen94guidance}.  Speed/surge model:
\begin{itemize}
\item Neglects added mass term based on Blanke comments.  Blanke suggests that the surge added mass term ``will typically be less then 5\%'' \cite{fossen94guidance}.
\item Neglects $r^2$ terms, as suggested by Blanke, based on assuming reasonabley low yaw-rate (< 10 degrees/s). 
\item Assumes linear+quadratic drag.  The Blanke model reported in \cite{fossen94guidance} includes only the quadratic term.  
\item Neglects the Coriolis terms based on assuing negligible sway velocity.  The experimental evidence from \cite{sonnenburg13modeling} and \cite{sonnenburg10control} indicate that there is a significant sideslip angle, hence a significant sway velocity.
\end{itemize}
Steering model
\begin{itemize}
\item Neglects added mass in surge and yaw.
\end{itemize}
Sway (sideslip) was not observable in experiments!  Used only GPS and heading for identification making making many of the estimated quantities unreliable or unobservable.

Thrust model neglects speed of advance, assumes thrust is independen of vessel speed.


\section{Maneuvering Models}
\subsection{Nonlinear Maneuvering Model based on Second-order Modulus Functions}
This model is uses second-order (linear and quadratic) terms for the dissapative terms. In this section we follow the notation and process detailed in \cite{fossen11handbook}, Chapter 7. The horiozontal-plane maneuvering model captures is formulated using state vector $\bm{\nu}=[u,v,r]^T$ where the velocities $u$, $v$ and $r$ are in the surge, sway and yaw directions respectively.  The velocities are considered to be relative to an irrotational constant ocean current.  The nonlinear maneuvering equations from \cite{fossen11handbook} are
\beqn
\underbrace{\bm{M}_{RB}\dot{\bm{\nu}}+\bm{C}_{RB}(\bm{\nu})\bm{\nu}}_\text{rigid-body forces} +
\underbrace{\bm{M}_A\dot{\bm{\nu}}_r + \bm{C}_A(\bm{\nu}_r)\bm{\nu}_r + 
\bm{D}(\bm{\nu}_r)\bm{\nu}_r}_\text{hydrodynamic forces}
= \bm{\tau}+\bm{\tau}_{wind}+\bm{\tau}_{waves}
\eeqn
where $\bm{\nu}_r$ is the velocity vector relative to an irrotational water current $\bm{\nu}_c$, i.e., $\bm{\nu}=\bm{\nu}_r+\bm{\nu}_c$.  The rigid body kinetics are represented by the rigid body mass $\bm{M}_{RB}$ 
\beqn
\bm{M}_{RB}= \left[ 
\begin{array}{ccc}
m & 0 & 0 \\
0 & m & m x_g \\
0 & m x_g & I_z 
\end{array} \right],
\eeqn
where $m$ is the mass of the vehicle, $I_z$ is the moment of inertia about the body-centered z-axis and $x_g$ is distance, along the x-axis, from the origin of the body-centered frame to the center of gravity of the vessel, and by the rigid body Coriolis-centripetal matrix,
\beqn
\bm{C}_{RB}(\bm{\nu})= \left[ 
\begin{array}{ccc}
0 & 0 & -m(x_gr+v) \\
0 & 0 & mu \\
m(x_gr+v) & -mu  & 0 
\end{array} \right].
\eeqn
Noting that $\bm{C}_{RB}(\bm{\nu})$ is skew-symmetric, i.e., $\bm{C}_{RB}(\bm{\nu})=-\bm{C}_{RB}^T(\bm{\nu})$.  The hydrodynamic effects are represented by the added mass matrix
\beqn
\bm{M}_{A}= \left[ 
\begin{array}{ccc}
-X_{\dot{u}} & 0 & 0 \\
0 & -Y_{\dot{v}} & -Y_{\dot{r}} \\
0 & -Y_{\dot{r}} & -N_{\dot{r}} 
\end{array} \right].
\eeqn
and the Coriolis-centripetal matrix for the added mass
\beqn
\bm{C}_{A}(\bm{\nu}_r)= \left[ 
\begin{array}{ccc}
0 & 0 & Y_{\dot{v}}v_r+Y_{\dot{r}}r \\
0 & 0 & -X_{\dot{u}}u_r\\
 -Y_{\dot{v}}v_r - Y_{\dot{r}}r& X_{\dot{u}}u_r & 0 
\end{array} \right].
\eeqn
It is worth noting that $\bm{C}_A$ includes the nonlinear Munk moment (see \cite{fossen11handbook} p.121).  Following \cite{fossen11handbook} the SNAME notation for the hydrodynamic derivatives.

The linear and quadratic drag terms
\beqn
\bm{D}(\bm{\nu}_r)= \left[ 
\begin{array}{ccc}
X_u + X_{u|u|}|u| & 0 & 0 \\
0 & Y_v + Y_{v|v|}|v| & Y_r+Y_{r|r|}|r|\\
0 & N_v + N_{v|v|}|v| & N_r+N_{r|r|}|r|
\end{array} \right].
\eeqn

Equivalently we can express the same model in non-matrix form, where the speed equation in the surge direction is 
\beqn
\underbrace{m \dot{u}}_\text{RB inertia}  
- \underbrace{m x_g r^2}_\text{RB centripetal}
- \underbrace{mvr}_\text{RB Coriolis}
=
\underbrace{X_{\dot{u}} \dot{u}_r}_\text{AM inertia}
+ \underbrace{Y_{\dot{v}}v_r r}_\text{AM Coriolis}
+ \underbrace{Y_{\dot{r}}r^2}_\text{AM centripetal}
+ \underbrace{X_u u + X_{u|u|}|u|u}_\text{Drag} 
+ \underbrace{\tau}_\text{Thrust}
\label{e:fullu}
\eeqn
and the coupled steering questions in the sway direction 
\beqn
\underbrace{m \dot{v} + m x_g \dot{r}}_\text{RB inertia}  
+ \underbrace{m ur}_\text{RB Coriolis}
- \underbrace{Y_{\dot{v}}\dot{v} - Y_{\dot{r}}\dot{r}}_\text{AM inertia}
=
+ \underbrace{Y_v v + Y_{v|v|}|v|v}_\text{Drag} 
+ \underbrace{Y_r r + Y_{r|r|}|r|r}_\text{Coupled Drag} 
\label{e:fullv}
\eeqn
and yaw directions 

\beqn
\underbrace{I_z \dot{r} + m x_g \dot{v}}_\text{RB inertia}  
+ \underbrace{m x_g ru}_\text{RB Coriolis}
- \underbrace{Y_{\dot{r}}\dot{v} - N_{\dot{r}}\dot{r}}_\text{AM inertia}
- \lefteqn{\overbrace{\phantom{Y_{\dot{v}}v_r u + X_{\dot{u}}u_r v}}^\text{Monk Moment}}
\underbrace{Y_{\dot{v}}v_r u + X_{\dot{u}}u_r v - Y_{\dot{r}}ru}_\text{AM Coriolis}
- \underbrace{N_v v + N_{v|v|}|v|v}_\text{Coupled Drag} 
+ \underbrace{N_r r + N_{r|r|}|r|r}_\text{Drag} 
= 
+ \underbrace{\tau}_\text{Torque}
\label{e:fullr}
\eeqn

\subsection{Nonlinear Model of Blanke}
As reported by \cite{fossen94guidance}, this model ``retains the most imoprtant terms for steering and propulsion''.  Formulating these equations for a rudder-less vessel, the speed equation becomes
\beqn
(m-X_{\dot{u}}) \dot{u} = X_{u|u|}|u|u + (m+X_{vr})vr + (mx_g+X_{rr})r^2 + \tau
\label{e:blankeu}
\eeqn
Comparing (\ref{e:fullu}) and (\ref{e:blankeu}), noting that $Y_{\dot{v}}$ from  (\ref{e:fullu}) is equivalent to $X_{vr}$ in (\ref{e:blankeu}),  we can see that the only difference between these models is that the full model in (\ref{e:fullu}) includes both linear and quadratic drag terms, while Blanke's model includes only a quadratic term.  The linear term is important because it forces the system to coverge to zero surge.  Futhermore, results of other experiments suggest that the drag is not purely quadratic.  

Blanke's steering model in sway, without rudder terms, is
\beqn
(m-Y_{\dot{v}}) \dot{v} + (m x_g - Y_{\dot{r}}) \dot{r}
=
-(m -Y_{ur})ur + Y_{uv}uv + Y_{v|v|}|v|v + Y_{|v|r}|v|r.
\label{e:blankev}
\eeqn
Comparing (\ref{e:fullv}) and (\ref{e:blankev}) we note the following differences
\begin{itemize}
\item The ``Coupled Drag'' term in (\ref{e:fullv}) is neglected in the Blanke model. \hl{What is the affect of neglecting these coupled drag terms? Appears faily commonly in simplified models.}
\item The linear component of cross-flow drag (\ref{e:fullv}) is neglected in the Blanke model
\item The following second-order cross-coupling terms are present in Blanke that are not represented in  (\ref{e:fullv})
  \begin{itemize}
    \item $Y_{ur}ur$ - which appears as an added mass Coriolis term
    \item $Y_{uv}uv$
    \item $Y_{|v|r}|v|r$
    \item \hl{What is the consequence of neglecting these? 
Ditto for the yaw eqn.}
  \end{itemize}
\end{itemize}
The steering equation for yaw is
\beqn
(m x_g - N_{\dot{v}})\dot{v} + (I_z - N_{\dot{r}})\dot{r}
=
-(m x_g - N_{ur})ur + N_{uv}uv + N_{v|v|}|v|v + N_{|v|r}|v|r
\label{e:blanker}
\eeqn
\hl{What is the yaw drag expressed as $N_{v|v|}|v|v$?  Seems like it would be $N_{r|r|}|r|r$}
Comparing (\ref{e:fullr}) and (\ref{e:blanker}), we note that the Monk moment has been collapsed into a single term $N_{uv}uv$ in (\ref{e:blanker}).  We note the following differences
\begin{itemize}
\item The ``Coupled Drag'' term in (\ref{e:fullr}) is neglected in the Blanke model
\item The linear component of yaw drag (\ref{e:fullv}) is neglected in the Blanke model
\item The second-order cross-coupling term $N_{|v|r}|v|r$ is present in Blanke model and not represented in\eref{e:fullr}.
\end{itemize}

\subsection{Simplified Model of Blanke from \cite{caccia08practical}}
In \cite{caccia08practical} the Blanke model is used, with the addition of linear drag terms, and the following assumptions are made:
\begin{itemize}
\item Speed: Surge
  \begin{itemize}
  \item Added mass in surge ($X_{\dot{u}}$) is negligible - less than 5\% of vessel mass.  This seem reasoable, but is not really necessary since the mass and added mass are lumped together in the identification.
  \item The Coriolis terms $(m+X_{vr})vr$, both rigid body and added mass contributions, are negligible because the sway speed is negligible.  \hl{This would seem to limit the sideslip.}  The experiments in \cite{sonnenburg10control}suggest the sideslip would be an important component for this type of vehicle.
  \item The centripetal terms $(mx_g+X_{rr})r^2$ are neglected on the assumption of a low turn rate.  This is probably reasonable?
  \end{itemize}
\item Steering: Sway and Yaw
  \begin{itemize}
  \item Added mass in both directions is neglected.  Again, probably not necessary since the identification estimates the combination of rigid body and added mass.
  \item Although the author's don't make it explicit(!), they neglect the rigid body and added mass coupling terms  \hl{I believe this is equivalent to assuming that the principle axes of inertia for the vessel (the body-centered coordinate frame) are co-located with the center of mass of the vessel.}
    \begin{itemize}
    \item $(m x_g - Y_{\dot{r}})\dot{r}$ in the sway equation
    \item $(m x_g - N_{\dot{v}})\dot{v}$ in the yaw equation
    \end{itemize}
  \item The ``coupled drag terms are neglected because $v$ and $r$ are typically small''.  We can only assume they are refering to the folloing terms - \hl{Are these correctly identified as ``coupling drag terms''?  Are they negligible?}
    \begin{itemize}
    \item $Y_{uv}uv$ and $Y_{|v|r}|v|r$ in sway
    \item 
    \end{itemize}
  \end{itemize}
\end{itemize}
For our vessel this would be equivalent to the following steering equations
\beqn
(m) \dot{v}
=
-(m -Y_{ur})ur + Y_v v + Y_{v|v|}|v|v .
\label{e:blanke2v}
\eeqn
\beqn
I_z\dot{r}
=
-(m x_g - N_{ur})ur + N_{uv}uv + N_{v|v|}|v|v + N_{|v|r}|v|r
\label{e:blanke2r}
\eeqn
\subsection{Thrust Model}
Two options:

Assume thrust is independent of speed (as done in the Caccia papers).

Or assume and unknown, linear decrease in thrust with speed.
\[
T = T_o (1-au)
\]
where $a$ is the linear speed reduction
\subsection{Speed model}
Consider the surge state of the model above where
\beqn
\underbrace{m \dot{u}}_\text{RB inertia}  
- \underbrace{m x_g r^2}_\text{RB centripetal}
- \underbrace{mvu}_\text{RB Coriolis}
=
\underbrace{X_{\dot{u}} \dot{u}}_\text{AM inertia}
+ \underbrace{Y_{\dot{v}}v_r r}_\text{AM Coriolis}
+ \underbrace{Y_{\dot{r}}r^2}_\text{AM centripetal}
+ \underbrace{X_u u + X_{u|u|}|u|u}_\text{Drag} 
+ \underbrace{T}_\text{Thrust}
\eeqn
Following \cite{caccia08practical}, based on \cite{fossen94guidance}, we neglect the second-order centripetal terms
\beqn
m \dot{u}
- mvu
=
X_{\dot{u}} \dot{u}
+ Y_{\dot{v}}v_r r
+ X_u u + X_{u|u|}|u|u
+ T
\eeqn

For steady state forward motion ($\dot{u}=v=r=0$) in stationary water ($v_r=0$)
\beqn
0 =
+ X_u u + X_{u|u|}|u|u
+ T
\eeqn
We can estimate $X_u$ and $X_{u|u}$ from steady state forward motion trials with known thrust input by testing at a series of known forward speeds and measuring $u$.

Considering forward-only acceleration
\beqn
m \dot{u}
=
X_{\dot{u}} \dot{u}
+ X_u u + X_{u|u|}|u|u
+ T
\eeqn
we can identify the added mass ($X_{\dot{u}}$) by either estimating the initial acceleration (see \cite{sonnenburg10control}) or by examing the 'time constant' for such tests.

This leaves the coefficient $Y_{\dot{v}}$, related to the added-mass Coriolis force, as the single unknown.

\subsection{Steering Model}


\section{Model Identification Tests}
\subsection{Physical Measurements}
\begin{itemize}
\item Measure the mass ($m$) directly.  
\item Measure the moment of inertia ($I_z$) using a bifilar pendulum.
\end{itemize}

\subsection{Thrust Characterization}
Bollard tests in the tank to measure thrust force (at zero velocity) as a function of motor command.

\subsection{Steady-State Tests}
\begin{itemize}
\item Surge: Measure the steady-state speed at a variety of thrust inputs to identify the drag terms.
\item Yaw: Measure the steady-state yaw rate at variety of torque inputs to identify the yaw drag terms.
\end{itemize}


\subsection{Open-Loop Dynamic Tests}
\begin{itemize}
\item Surge: Measure step response to forward thruste (with heading control?) to estimate added surge mass.
\item Yaw: Measure step response to torque to estimate added mass/ineriat in yaw.
\end{itemize}

\subsection{Closed-Loop Dynamic Tests}

%\section{Ship Dynamics}
%Following \cite{fossen94guidance}, Ch 5.


%\section{Acknowledgments}

% standard IEEE bibliography style from:
% http://www.ctan.org/tex-archive/macros/latex/contrib/supported/IEEEtran/bibtex
%\bibliographystyle{../latex_ieee/IEEEtran}
\bibliographystyle{apalike}
% argument is your BibTeX string definitions and bibliography database(s)
\bibliography{../bibtexdatabase/bbing_master}
%\bibliography{./latex_ieee/IEEEabrv}

% if you will not have a photo at all:

% that's all folks

\end{document}


