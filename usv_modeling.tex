%%%%%%%%%%%%%%%%%%%%%%%%%%%%%%%%%%%%%%%%%%%%%%%
% latex_template_ieee_bbing.tex
%
% 28.02.2008  bsb Created 
%%%%%%%%%%%%%%%%%%%%%%%%%%%%%%%%%%%%%%%%%%%%%%%%%

% Final - 2 column style
%\documentclass[10pt,final,journal]{../latexlib/latex_ieee/IEEEtran}
% Draft - single column style
\documentclass[11pt,draftcls,journal,onecolumn]{../latexlib/latex_ieee/IEEEtran}

% If the IEEEtran.cls has not been installed into the LaTeX system files, 
% manually specify the path to it:
% \documentclass[conference]{../sty/IEEEtran} 

% Gives LaTeX2e the abilsity to do double column
%% correct bad hyphenation here
\hyphenation{op-tical net-works semi-conduc-tor IEEEtran}

% From thesis main (bbing)
\usepackage{amssymb,longtable,dcolumn}

% to use pdflatex
% Standard bbing packages
\usepackage{cite}      % Written by Donald Arseneau
\usepackage{graphicx}  % Written by David Carlisle and Sebastian Rahtz
\usepackage{url}       % Written by Donald Arseneau
\usepackage{amssymb,longtable,dcolumn}
\usepackage{bm}
%\usepackage{subfigure}
%\usepackage{stfloats}  % Written by Sigitas Tolusis
\usepackage[caption=false,font=footnotesize]{subfig}
%\usepackage{fixltx2e}
\usepackage{colortbl}
\usepackage{multirow}
\usepackage{amsmath}
\usepackage{units}
\usepackage{../latexlib/latex_cmds/math_bbing}
\usepackage{acronym}
\usepackage{csvsimple}
\usepackage{../latexlib/latex_cmds/my_acronyms}
\usepackage{color,soul}

\begin{document}

\newtheorem{remark}{Remark}
\renewcommand{\theremark}{\unskip}


\input{../latexlib/latex_cmds/Commands}  % shortcuts to thesis stuff
\input{../latexlib/latex_cmds/defs}
%\include{./latex_cmds/Commands}  % shortcuts to thesis stuff
%\include{./latex_cmds/defs}


% set the figure default size
\newcommand{\SF}{0.2}
\newcommand{\SFb}{0.45}
\newcommand{\SFPic}{0.45}
\newcommand{\SFPlot}{0.45}
\newcommand{\SFc}{0.25}
% Just a lazy way of setting the figure width (percentage of text width)
% 0.7 works well for 1 column
% 0.4 works well for 2 column
\newcommand{\FigWidth}{\SFb}

% Use this one for the draft version
\newcommand{\scaleOneTwo}[2] {\scalebox{#1}}
% Use this one for the two column version
%\newcommand{\scaleOneTwo}[2] {\scalebox{#2}}

% Graphics for this paper
\graphicspath{{./figs/}}

% paper title
%\title{Towards an Experimentally Validated Plume Model to Support Robotic Plume Characterization}
\title{USV Modeling}

% author names and affiliations
% use a multiple column layout for up to three different
% affiliations
\author{Brian~Bingham$^{1}$% <-this % stops a space
\thanks{$^{1}$ Brian~Bingham is with the Department of Mechanical and Aerospace Engineering, Naval Postgraduate School, Monterey, CA 93950, USA. {\tt\small bbingham@nps.edu}}%
}

% make the title area
\maketitle

\begin{abstract}
Abstract
\end{abstract}
% no keywords

\IEEEpeerreviewmaketitle

\section{Introduction}

\section{Background}

\cite{sonnenburg03modeling} examines model for USV with steerable outboard motor (vectored thrust) where sideslip is a major concern.  Uses notation from \cite{fossen94guidance}

\section{Maneuvering Model}
In this section we follow the notation and process detailed in \cite{fossen11handbook}. The horiozontal-plane maneuvering model captures is formulated using state vector $\bm{\nu}=[u,v,r]^T$ where the velocities $u$, $v$ and $r$ are in the surge, sway and yaw directions respectively.  The velocities are considered to be relative to an irrotational constant ocean current.
\beqn
\bm{M}_{RB}\dot{\bm{\nu}} + \bm{M}_A\dot{\bm{\nu}}_r + 
\bm{C}_{RB}(\bm{\nu})\bm{\nu} + \bm{C}_A(\bm{\nu}_r)\bm{\nu}_r + 
\bm{D}(\bm{\nu}_r)\bm{\nu}_r
= \bm{\tau}+\bm{\tau}_{wind}+\bm{\tau}_{waver}
\eeqn
where $\bm{\nu}_r$ is the velocity vector relative to an irrotational water current $\bm{\nu}_c$, i.e., $\bm{\nu}=\bm{\nu}_r+\bm{\nu}_c$.  The rigid body kinetics are represented by the rigid body mass $\bm{M}_{RB}$ 
\beqn
\bm{M}_{RB}= \left[ 
\begin{array}{ccc}
m & 0 & 0 \\
0 & m & m x_g \\
0 & m x_g & I_z 
\end{array} \right],
\eeqn
where $m$ is the mass of the vehicle, $I_z$ is the moment of inertia about the body-centered z-axis and $x_g$ is distance, along the x-axis, from the origin of the body-centered frame to the center of gravity of the vessel, and by the rigid body Coriolis-centripetal matrix,
\beqn
\bm{C}_{RB}(\bm{\nu})= \left[ 
\begin{array}{ccc}
0 & 0 & -m(x_gr+v) \\
0 & 0 & mu \\
m(x_gr+v) & -mu  & 0 
\end{array} \right].
\eeqn
Noting that $\bm{C}_{RB}(\bm{\nu})$ is skew-symmetric, i.e., $\bm{C}_{RB}(\bm{\nu})=-\bm{C}_{RB}^T(\bm{\nu})$.  The hydrodynamic effects are represented by the added mass matrix
\beqn
\bm{M}_{A}= \left[ 
\begin{array}{ccc}
-X_{\dot{u}} & 0 & 0 \\
0 & -Y_{\dot{v}} & -Y_{\dot{r}} \\
0 & -Y_{\dot{r}} & -N_{\dot{r}} 
\end{array} \right].
\eeqn
and the Coriolis-centripetal matrix for the added mass
\beqn
\bm{C}_{A}(\bm{\nu}_r)= \left[ 
\begin{array}{ccc}
0 & 0 & Y_{\dot{v}}v_r+Y_{\dot{r}}r \\
0 & 0 & -X_{\dot{u}}u_r\\
 -Y_{\dot{v}}v_r - Y_{\dot{r}}r& X_{\dot{u}}u_r & 0 
\end{array} \right].
\eeqn
It is worth noting that $\bm{C}_A$ includes the nonlinear Munk moment.  Following \cite{fosson11handbook} the SNAME notation for the hydrodynamic derivatives.

Use \cite{sonnenburg10control} to get linear + quadratic

\section{Acknowledgments}

% standard IEEE bibliography style from:
% http://www.ctan.org/tex-archive/macros/latex/contrib/supported/IEEEtran/bibtex
%\bibliographystyle{../latex_ieee/IEEEtran}
\bibliographystyle{apalike}
% argument is your BibTeX string definitions and bibliography database(s)
\bibliography{../bibtexdatabase/bbing_master}
%\bibliography{./latex_ieee/IEEEabrv}

% if you will not have a photo at all:

% that's all folks

\end{document}


